% This must be in the first 5 lines to tell arXiv to use pdfLaTeX, which is strongly recommended.
\pdfoutput=1
% In particular, the hyperref package requires pdfLaTeX in order to break URLs across lines.

\documentclass[11pt]{article}

\usepackage[a4paper, total={6in, 8in}]{geometry}
\usepackage{lineno}
\linenumbers

\usepackage{caption,subcaption}

\usepackage{mystyle}

\usetikzlibrary{calc,patterns,angles,quotes}    
\usetikzlibrary{decorations.pathmorphing}
\tikzset{snake it/.style={decorate, decoration=snake}}

\usepackage{natbib}
\bibliographystyle{abbrvnat}
\usepackage{hyperref}

% Standard package includes
\usepackage{times}
\usepackage{latexsym}

% For proper rendering and hyphenation of words containing Latin characters (including in bib files)
\usepackage[T1]{fontenc}
% For Vietnamese characters
% \usepackage[T5]{fontenc}
% See https://www.latex-project.org/help/documentation/encguide.pdf for other character sets

% This assumes your files are encoded as UTF8
\usepackage[utf8]{inputenc}


\title{Network Optimization}

\author{Justin Chiu \\
  Cornell Tech \\
  \texttt{jtc257@cornell.edu}}

\begin{document}
\maketitle
\begin{abstract}
None
\end{abstract}

\section{Network Traffic Problem}
In this section, we consider the problem of optimizing network traffic.
A network is a graph $G = (V,E)$ with vertices $v_i\in V$ and edges
$e_{ij}\in E = V\times V$.
We would like to maximize the total traffic through a series of $K$ paths across the network.
A path is a sequence of edges $p_{st}^k = (e_{sv_1},e_{v_1v_2},\ldots,e_{v_mt})$
from source vertex $s$ to target vertex $t$.
A path contributes a constant amount of traffic $x_p$ to each included edge $e\in p$.
We denote the set of paths that pass through a particular edge by
$\pi(e) = \set{p \mid e \in p}$.
We additionally have the following constraints: traffic must be nonnegative $x_p \geq 0$,
each edge has a capacity constraint so that the total traffic on that
edge cannot exceed $ c_e \ge \sum_{p\in\pi(e)} x_p $,
and we should not exceed the demand for traffic on a particular path $x_p \leq d_p$.

This yields the following optimization problem:
\begin{equation}
\label{eqn:network-opt}
\begin{aligned}
\textrm{maximize } \quad & \sum_{p} x_p \textrm{ or } \sum_{e} \sum_{p\in\pi(e)}x_{p}\\
\textrm{subject to } \quad &\sum_{p\in\pi(e)}x_p \le c_e, \forall e\\
&x_p \le d_p, \forall p\\
& x_p \geq 0, \forall p.
\end{aligned}
\end{equation}
The first objective assigns equal weight to each path, while the second
objective weights paths based on length.
We focus on the first objective, since the goal is to maximize the amount of traffic
each customer receives (represented as a path) rather than the congestion
of the network.

While this form is compact, we would like to apply the ADMM algorithm,
which applies to problems of the form
\begin{equation}
\label{eqn:admm}
\begin{aligned}
\textrm{minimize } \quad & f(x) + g(z)\\
\textrm{subject to } \quad & Ax + Bz = c.
\end{aligned}
\end{equation}

To massage the problem in Eqn.~\ref{eqn:network-opt} into the correct form,
we will introduce two sets of new variables: slack variables $s_e$ for each edge\footnote{
The slack variable replaces the inequality constraints $c_e - \sum_p x_p \ge 0$
with equality constraints $c_e - \sum_p x_p - s_e = 0$.
}
and decoupled edge weights $z_{pe}$ for each edge and path.

The new problem is given by
\begin{equation}
\label{eqn:network-opt-std}
\begin{aligned}
\textrm{minimize } \quad & f(x) + g(z) + h(s)\\
\textrm{subject to } \quad &c_e - \sum_{p\in\pi(e)}z_{pe} - s_{e} = 0, \forall e\\
& d_p - x_p - s_p = 0, \forall p\\
&x_p - z_{pe} = 0, \forall e, \forall p \in \pi(e)\\
&x,s,z \succeq 0,
\end{aligned}
\end{equation}
where $f(x) = -\sum_p x_p$,
$g(z) = \sum_{pe} \bar\delta(z_{pe} \ge 0)$,
$h(s) = \sum_{e} \bar\delta(s_{e} \ge 0)+ \sum_{p} \bar\delta(s_{p} \ge 0) $.
We use the delta notation to indicate a function
$$
\bar\delta(b) = \begin{cases}
    \infty & \textrm{if condition } b \textrm{ does not hold},\\
    0 & \textrm{o.w.}
\end{cases}
$$
This problem is decomposable along each path (for $x$) or edge (for $z$) as follows:
\begin{equation}
\label{eqn:network-opt-sep}
\begin{aligned}
\textrm{minimize } \quad & -\sum_p x_p
    + \sum_e\sum_{p \in \pi(e)} \bar\delta(z_{pe})
    + \sum_e \bar\delta(s_{e}) + \sum_p \bar\delta(s_p)\\
\textrm{subject to } \quad &c_e - \sum_{p\in\pi(e)}z_{pe} - s_{e} = 0, \forall e\\
& d_p - x_p - s_p = 0, \forall p\\
&x_p - z_{pe} = 0, \forall e, \forall p \in \pi(e)\\
&x,s,z \succeq 0.
\end{aligned}
\end{equation}

The augmented Lagrangian for this problem is
\begin{equation}
    \label{eqn:network-lagrangian}
    \mcL_\rho(x,z,s,\lambda) =-\sum_p x_p + \sum_e\sum_{p\in\pi(e)} \bar\delta(z_{pe})
        + \sum_{e} \bar\delta(s_{e})
        + \sum_{p} \bar\delta(s_{p})
        + \lambda^\top F(x,z,s)
        + (\rho/2)\| F(x,z,s) \|_2^2,
\end{equation}
where
$$F(x,z,s) =
\begin{bmatrix}
F_1 \\
F_2 \\
F_3
\end{bmatrix}
$$
with $F_1 \in \R^{|E|}, F_2\in\R^{K|E|}$
($|E|$ the number of edges and $K$ the number of paths).
Each subvector is given by
$$
    [F_1]_e = c_e - \sum_{p\in\pi(e)}z_{pe} - s_{e},
    [F_2]_{p} = d_p - x_p - s_p,
    [F_3]_{pe} = x_p - z_{pe},
$$
The elements of $\lambda$ corresponding to $F_1$ are $\lambda_{1,e}$,
$F_2$ are $\lambda_{2,p}$, and $F_3$ are $\lambda_{3,pe}$.

The ADMM updates are as follows:
\begin{equation}
    \begin{aligned}
        x^{k+1} := \argmin_{x} \mcL_\rho(x,z^k,s^{k},\lambda^k)\\
        z^{k+1} := \argmin_{z} \mcL_\rho(x^{k+1},z,s^{k},\lambda^k)\\
        s^{k+1} := \argmin_{s} \mcL_\rho(x^{k+1},z^{k+1},s,\lambda^k)\\
        \lambda^{k+1} := \lambda^k + \rho(F(x^{k+1},z^{k+1},s^{k+1})).
    \end{aligned}
\end{equation}
We can compute $\argmin_{x} \mcL_\rho(x,z^k,s^k,\lambda^k)$ by restricting our
attention to terms of $\mcL_\rho$ involving $x$ and setting the derivative equal to 0:
\begin{align*}
0
&= \nabla_{x_p}(-x_p + \lambda_{2,p}(d_p - x_p - s_p) + \sum_{e\in p}\lambda_{3,pe}(x_p - z_{pe})
+ (\rho/2)((d_p - x_p - s_p) + \sum_{e \in p}(x_p - z_{pe})^2))\\
&= -1 -\lambda_{2,p} + \sum_{e \in p}\lambda_{3,pe} + \rho\sum_{e \in p} (x_p - z_{pe})\\
x_p &= \max(0,\frac{1 + \lambda_{2,p} - \sum_{e \in p}\lambda_{3,pe} + \rho \sum_{e\in p} z_{pe}}
{|p|\rho}).
\end{align*}

We perform a similar computation for $\argmin_{z} \mcL_\rho(x^{k+1},z,s^k,\lambda^k)$:
\begin{align*}
0
&= \nabla_{z_{pe}}\bar\delta(z_{pe}) + \lambda_{1,e}(c_e - \sum_{p'\in\pi(e)} z_{p'e} - s_{e})
    + \lambda_{3,pe}(x_p - z_{pe})\\
& \qquad + (\rho/2)((c_e - \sum_{p'\in\pi(e)} z_{p'e} - s_e)^2 + (x_p - z_{pe})^2)\\
&= -\lambda_{1,e} - \lambda_{3,pe}
    + \rho(-c_e + \sum_{p'\in\pi(e)} z_{p'e} + s_e  + z_{pe} - x_p).
\end{align*}
This gives us a system of equations for each edge $0 = A_ez_e + b_e$
allowing us to solve for $z_e = -A_e^{-1}b_e$.
Let $P_e= |\pi(e)|$. We then have
\begin{align*}
A_e &= \mathbf{1}_{P_e\times P_e} + I_{P_e\times P_e}\\
[b_e]_p &= -\lambda_{1,e} - \lambda_{3,pe}
    + \rho(-c_e + s_e - x_p).
\end{align*}

Then, for $\argmin_{s} \mcL_\rho(x^{k+1},z^{k+1},s,y^k)$:
\begin{align*}
0
&= \nabla_{s_{e}}\bar\delta(s_{e}) + \lambda_{1,e}(c_e - \sum_{p\in\pi(e)} z_{pe} - s_{e})
    + (\rho/2)((c_e - \sum_{p\in\pi(e)} z_{pe} - s_{e})^2)\\
&= -\lambda_{1,e} + \rho(-c_e + \sum_{p\in\pi(e)} z_{pe} + s_e)\\
s_{e} &= \max(0,\frac{\lambda_{1,e} + \rho(c_e - \sum_{p\in\pi(e)}z_{pe})}{\rho}),
\end{align*}
and
\begin{align*}
0
&= \nabla_{s_{p}}\bar\delta(s_{p}) + \lambda_{2,p}(d_p - x_p - s_p)
    + (\rho/2)((d_p - x_p - s_{p})^2)\\
&= -\lambda_{2,p} + \rho(-d_p + x_p + s_p)\\
s_{p} &= \max(0,\frac{\lambda_{2,p} + \rho(d_p - x_p)}{\rho}).
\end{align*}

\end{document}
