% This must be in the first 5 lines to tell arXiv to use pdfLaTeX, which is strongly recommended.
\pdfoutput=1
% In particular, the hyperref package requires pdfLaTeX in order to break URLs across lines.

\documentclass[11pt]{article}

\usepackage[a4paper, total={6in, 8in}]{geometry}
\usepackage{lineno}
\linenumbers

\usepackage{caption,subcaption}

\usepackage{mystyle}

\usetikzlibrary{calc,patterns,angles,quotes}    
\usetikzlibrary{decorations.pathmorphing}
\tikzset{snake it/.style={decorate, decoration=snake}}

\usepackage{natbib}
\bibliographystyle{abbrvnat}
\usepackage{hyperref}

% Standard package includes
\usepackage{times}
\usepackage{latexsym}

% For proper rendering and hyphenation of words containing Latin characters (including in bib files)
\usepackage[T1]{fontenc}
% For Vietnamese characters
% \usepackage[T5]{fontenc}
% See https://www.latex-project.org/help/documentation/encguide.pdf for other character sets

% This assumes your files are encoded as UTF8
\usepackage[utf8]{inputenc}


\title{Network Stuff}

\author{Justin Chiu \\
  Cornell Tech \\
  \texttt{jtc257@cornell.edu}}

\begin{document}
\maketitle
\begin{abstract}
None
\end{abstract}

\section{Network Traffic Problem}
In this section, we consider the problem of optimizing network traffic.
A network is a graph $G = (V,E)$ with vertices $v_i\in V$ and edges
$e_{ij}\in E = V\times V$.
We would like to maximize the total traffic through a series of $K$ paths across the network.
A path is a sequence of edges $p_{st}^k = (e_{sv_1},e_{v_1v_2},\ldots,e_{v_mt})$
from source vertex $s$ to target vertex $t$.
A path contributes a constant amount of traffic $x_p$ to each included edge $e\in p$.
We denote the set of paths that pass through a particular edge by
$\pi(e) = \set{p \mid e \in p}$.
We additionally have the following constraints: traffic must be nonnegative $x_p \geq 0$,
and each edge has a capacity constraint that the total traffic on that
edge cannot exceed $\sum_{p\in\pi(e)} x_p \le c_e$.

This yields the following optimization problem:
\begin{equation}
\label{eqn:network-opt}
\begin{aligned}
\textrm{maximize } \quad & \sum_{p} x_p\\
\textrm{subject to } &\sum_{p\in\pi(e)}x_p \le c_e, \forall e\\
& x_p \geq 0, \forall p.
\end{aligned}
\end{equation}


\end{document}
