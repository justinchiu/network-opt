% This must be in the first 5 lines to tell arXiv to use pdfLaTeX, which is strongly recommended.
\pdfoutput=1
% In particular, the hyperref package requires pdfLaTeX in order to break URLs across lines.

\documentclass[11pt]{article}

\usepackage[a4paper, total={6in, 8in}]{geometry}
\usepackage{lineno}
\linenumbers

\usepackage{caption,subcaption}

\usepackage{mystyle}

\usetikzlibrary{calc,patterns,angles,quotes}    
\usetikzlibrary{decorations.pathmorphing}
\tikzset{snake it/.style={decorate, decoration=snake}}

\usepackage{natbib}
\bibliographystyle{abbrvnat}
\usepackage{hyperref}

% Standard package includes
\usepackage{times}
\usepackage{latexsym}

% For proper rendering and hyphenation of words containing Latin characters (including in bib files)
\usepackage[T1]{fontenc}
% For Vietnamese characters
% \usepackage[T5]{fontenc}
% See https://www.latex-project.org/help/documentation/encguide.pdf for other character sets

% This assumes your files are encoded as UTF8
\usepackage[utf8]{inputenc}


\title{Network Stuff}

\author{Justin Chiu \\
  Cornell Tech \\
  \texttt{jtc257@cornell.edu}}

\begin{document}
\maketitle
\begin{abstract}
None
\end{abstract}

\section{Network Traffic Problem}
In this section, we consider the problem of optimizing network traffic.
A network is a graph $G = (V,E)$ with vertices $v_i\in V$ and edges
$e_{ij}\in E = V\times V$.
We would like to maximize the total traffic through a series of $K$ paths across the network.
A path is a sequence of edges $p_{st}^k = (e_{sv_1},e_{v_1v_2},\ldots,e_{v_mt})$
from source vertex $s$ to target vertex $t$.
A path contributes a constant amount of traffic $x_p$ to each included edge $e\in p$.
We denote the set of paths that pass through a particular edge by
$\pi(e) = \set{p \mid e \in p}$.
We additionally have the following constraints: traffic must be nonnegative $x_p \geq 0$,
and each edge has a capacity constraint that the total traffic on that
edge cannot exceed $\sum_{p\in\pi(e)} x_p \le c_e$.

This yields the following optimization problem:
\begin{equation}
\label{eqn:network-opt}
\begin{aligned}
\textrm{maximize } \quad & \sum_{p} x_p \textrm{ or } \sum_{e} \sum_{p\in\pi(e)}x_{p}\\
\textrm{subject to } \quad &\sum_{p\in\pi(e)}x_p \le c_e, \forall e\\
& x_p \geq 0, \forall p.
\end{aligned}
\end{equation}
The first objective assigns equal weight to each path, while the second
objective weights paths based on length.

While this form is compact, we would like to apply the ADMM algorithm,
which applies to problems of the form
\begin{equation}
\label{eqn:admm}
\begin{aligned}
\textrm{minimize } \quad & f(x) + g(z)\\
\textrm{subject to } \quad & Ax + Bz = c.
\end{aligned}
\end{equation}

To massage the problem in Eqn.~\ref{eqn:network-opt} into the correct form,
we will assume the first (second) objective, which assigns (un)equal weight to each path.
We will introduce two sets of new variables, slack variables $s_e$ for each edge
(to replace the inequality constraints)
and edge weights $z_{pe}$ for each edge and path.

The new problem is given by
\begin{equation}
\label{eqn:network-opt-admm}
\begin{aligned}
\textrm{minimize } \quad & f(x) + g(z) + h(s)\\
\textrm{subject to } \quad &c_e - \sum_{p\in\pi(e)}z_p - s = 0, \forall e\\
&x_p - z_{pe} = 0, \forall e, \forall p \in \pi(e),
\end{aligned}
\end{equation}
where $f(x) = -\sum_p x_p$\footnote{
or $-\sum_e \sum_{p\in\pi(e)}x_p$},
$g(z) = \delta(z_{ep} < 0)$, $h(s) = \delta(s_e < 0)$.
We use the delta notation to indicate a function
$$
\delta(b) = \begin{cases}
    \infty & \textrm{if condition } b \textrm{ holds},\\
    0 & \textrm{ow}
\end{cases}
$$
This can then be expressed in matrix form as follows:


\end{document}
