% This must be in the first 5 lines to tell arXiv to use pdfLaTeX, which is strongly recommended.
\pdfoutput=1
% In particular, the hyperref package requires pdfLaTeX in order to break URLs across lines.

\documentclass[11pt]{article}

\usepackage[a4paper, total={6in, 8in}]{geometry}
\usepackage{lineno}
\linenumbers

\usepackage{caption,subcaption}

\usepackage{mystyle}

\usetikzlibrary{calc,patterns,angles,quotes}    
\usetikzlibrary{decorations.pathmorphing}
\tikzset{snake it/.style={decorate, decoration=snake}}

\usepackage{natbib}
\bibliographystyle{abbrvnat}
\usepackage{hyperref}

% Standard package includes
\usepackage{times}
\usepackage{latexsym}

% For proper rendering and hyphenation of words containing Latin characters (including in bib files)
\usepackage[T1]{fontenc}
% For Vietnamese characters
% \usepackage[T5]{fontenc}
% See https://www.latex-project.org/help/documentation/encguide.pdf for other character sets

% This assumes your files are encoded as UTF8
\usepackage[utf8]{inputenc}


\title{Network Optimization}

\author{Justin Chiu \\
  Cornell Tech \\
  \texttt{jtc257@cornell.edu}}

\begin{document}
\maketitle
\begin{abstract}
None
\end{abstract}

\section{Network Traffic Problem}
We consider the problem of maximizing network traffic.
A network is a graph $G = (V,E)$ with vertices $v\in V$ and edges
$e = (i,j)\in E = V\times V$.
Each edge $e$ has an associated capacity limit $c_e$.
We get a set of requests,
with each request $r$ represented as source and target pairs $(s,t)$,\footnote{
We assume there is only one unique request for each source and target pair.
}
plus a traffic demand $d_r$,
and the set of all paths $R$.
We would like to fulfill the demand for each request as much as possible,
by splitting traffic across the $K_r$ valid paths
$p_{rk} = ((s,a), (a,b), \ldots, (z,t))$ from $s$ to $t$:
$\sum_{p\in r} x_p \le d_r$, where we should not exceed the request's demand
(constraint [1]).
We will say that $p\in r$ if $p$ is a valid path from $s\to t$.
We denote the set of all paths $P$.
Additionally, we must ensure that edge traffic constraints hold.
Multiple paths, as well as multiple requests, may result in overlapping traffic
across particular edges, resulting in constraint [2]:
$\sum_{p\in\pi(e)} x_{p} \le c_e$,
where $\pi(e) = \set{p \mid e \in p}$.

This yields the following optimization problem:
\begin{equation}
\label{eqn:network-opt}
\begin{aligned}
\textrm{minimize } \quad & -\sum_r \sum_{p\in r} x_p\\
\textrm{subject to } \quad
&\sum_{p \in r}x_p \le d_r, \forall r & [1]\\
&\sum_{p\in\pi(e)} x_p \le c_e, \forall e & [2]\\
& x_p \geq 0, \forall p.
\end{aligned}
\end{equation}

While this form is compact, we would like the problem to
separate over paths. This is not possible in Eqn.~\ref{eqn:network-opt}
because constraint [2] couples the traffic in paths.
In order to decouple this constraint, we will transform the problem by adding dummy variables.
Afterwards, we will introduce slack variables to turn each of the resulting inequality constraints
into equality constraints, which makes it more straightforward to apply the ADMM algorithm.

We add one dummy variable $z_{pe}$ for each combination of path and edge.\footnote{
There are $K_r$ paths for each request $r$, resulting in
$|E|\sum_r\sum_{K_r}$ dummy variables.}
We then replace the $x_p$ variables constraint [2] with dummy variables $z_{pe}$,
resulting in constraint [3].
We also for $x_p = z_{pe}$ as constraint [4].
This results in the following problem:
\begin{equation}
\label{eqn:network-opt-sep}
\begin{aligned}
\textrm{minimize } \quad & -\sum_r \sum_{p\in r} x_p\\
\textrm{subject to } \quad
&\sum_{p \in r}x_p \le d_r, \forall r & [1]\\
&\sum_{p\in\pi(e)} z_{pe} \le c_e, \forall e & [3]\\
&x_p - z_{pe} = 0, \forall e,p & [4]\\
& x_p \geq 0, \forall p \\
& z_{pe} \geq 0, \forall e,p.
\end{aligned}
\end{equation}

Finally, we add slack variables $s_{1r},s_{3e}$
to turn the inequality constraints into equality constraints.
\begin{equation}
\label{eqn:network-opt-sep-eq}
\begin{aligned}
\textrm{minimize } \quad & -\sum_r \sum_{p\in r} x_p\\
\textrm{subject to } \quad
&d_r - \sum_{p \in r}x_p - s_{1r} = 0, \forall r & [1]\\
&c_e - \sum_{p\in\pi(e)} z_{pe} - s_{3e}= 0, \forall e & [3]\\
&x_p - z_{pe} = 0, \forall e,p & [4]\\
& x_p \geq 0,
z_{pe} \geq 0,
s_{1r} \geq 0,
s_{2e} \geq 0.
\end{aligned}
\end{equation}

In order to write the Lagrangian for this problem,
we introduce Lagrange multipliers
$\lambda = (\lambda_{1},\lambda_{3},\lambda_{4})$
for constraints [1], [3], and [4] respectively,
where $\lambda_1\in\R^{|R|},\lambda_3\in\R^{|E|},\lambda_4\in\R^{|P||E|}$.
The Lagrangian is then given by
\begin{equation}
    \label{eqn:network-lagrangian-std}
    \mcL_\rho(x,z,s,\lambda) =-\sum_r\sum_{p\in r} x_p + \lambda^\top F(x,z,s),
\end{equation}
where $F(x,z,s) = (F_1, F_2,F_3)^\top$ and
$$
[F_1]_r = d_r - \sum_{p \in r}x_p - s_{1r},\quad
[F_3]_e = c_e - \sum_{p\in\pi(e)} z_{pe} - s_{3e},\quad
[F_4]_{pe} = x_p - z_{pe}.
$$
The augmented Lagrangian for this problem is
\begin{equation}
    \label{eqn:network-lagrangian}
    \mcL_\rho(x,z,s,\lambda) =-\sum_r\sum_{p\in r} x_p + \lambda^\top F(x,z,s)
        + (\rho/2)\| F(x,z,s) \|_2^2,
\end{equation}
where we have added a quadratic term based on $F$ and the hyperparameter $rho\in\R_+$.
This quadratic term is key to ADMM.

Given the augmented Lagrangian, the ADMM updates are as follows:
\begin{equation}
    \begin{aligned}
        x^{k+1} := \argmin_{x} \mcL_\rho(x,z^k,s^{k},\lambda^k)\\
        z^{k+1} := \argmin_{z} \mcL_\rho(x^{k+1},z,s^{k},\lambda^k)\\
        s^{k+1} := \argmin_{s} \mcL_\rho(x^{k+1},z^{k+1},s,\lambda^k)\\
        \lambda^{k+1} := \lambda^k + \rho(F(x^{k+1},z^{k+1},s^{k+1})).
    \end{aligned}
\end{equation}
We can compute $\argmin_{x} \mcL_\rho(x,z^k,s^k,\lambda^k)$ by restricting our
attention to terms of $\mcL_\rho$ involving each $x_p$ (for each path $p\in P$)
and setting the derivative equal to 0:
\begin{align*}
0
&= \nabla_{x_p}(-x_p + \lambda_{1r}(d_r - \sum_{p\in r}x_p - s_{1r})
    + \sum_{e\in p}\lambda_{4pe}(x_p - z_{pe})\\
    & \qquad + (\rho/2)((d_r - \sum_{p\in r}x_p - s_{1r})^2 + \sum_{e \in p}(x_p - z_{pe})^2))\\
&= -1 -\lambda_{1r} + \sum_{e \in p}\lambda_{4pe} + \rho(-d_r + \sum_{p\in r}x_p + s_{1r})
    +\rho\sum_{e \in p} (x_p - z_{pe})\\
x_p &= \max(0,\frac{1 + \lambda_{1r} - \sum_{e \in p}\lambda_{4pe}
    + \rho(d_r -\sum_{p'\in e\setminus p} x_{p'}- s_{1r} + \sum_{e\in p} z_{pe})}
{(1+|p|)\rho}).
\end{align*}

We perform a similar computation for $\argmin_{z} \mcL_\rho(x^{k+1},z,s^k,\lambda^k)$:
\begin{align*}
0
&= \nabla_{z_{pe}}\lambda_{3e}(c_e - \sum_{p'\in\pi(e)} z_{p'e} - s_{3e})
    + \lambda_{4pe}(x_p - z_{pe})\\
& \qquad + (\rho/2)((c_e - \sum_{p'\in\pi(e)} z_{p'e} - s_e)^2 + (x_p - z_{pe})^2)\\
&= -\lambda_{3e} - \lambda_{4pe}
    + \rho(-c_e + \sum_{p'\in\pi(e)} z_{p'e} + s_e  + z_{pe} - x_p).
\end{align*}
This gives us a system of equations for each edge $0 = A_ez_e + b_e$
allowing us to solve for $z_e = -A_e^{-1}b_e$.
Let $P_e= |\pi(e)|$. We then have
\begin{align*}
A_e &= \mathbf{1}_{P_e\times P_e} + I_{P_e\times P_e}\\
[b_e]_p &= -\lambda_{3e} - \lambda_{4pe}
    + \rho(-c_e + s_e - x_p).
\end{align*}

Then, for $\argmin_{s_1} \mcL_\rho(x^{k+1},z^{k+1},s,y^k)$:
\begin{align*}
0
&= \nabla_{s_{1r}}\lambda_{1r}(d_r - \sum_{p'\in r}x_{p'} - s_{1p})
    + (\rho/2)((d_r - \sum_{p'\in r}x_{p'} - s_{1p})^2)\\
&= -\lambda_{1r} + \rho(-d_p + \sum_{p'\in r}x_{p'} + s_p)\\
s_{p} &= \max(0,\frac{\lambda_{1r} + \rho(d_p - \sum_{p'\in r}x_{p'})}{\rho}),
\end{align*}
and for $\argmin_{s_3} \mcL_\rho(x^{k+1},z^{k+1},s,y^k)$:
\begin{align*}
0
&= \nabla_{s_{3e}} \lambda_{3e}(c_e - \sum_{p\in\pi(e)} z_{pe} - s_{3e})
    + (\rho/2)((c_e - \sum_{p\in\pi(e)} z_{pe} - s_{3e})^2)\\
&= -\lambda_{3e} + \rho(-c_e + \sum_{p\in\pi(e)} z_{pe} + s_{3e})\\
s_{3e} &= \max(0,\frac{\lambda_{3e} + \rho(c_e - \sum_{p\in\pi(e)}z_{pe})}{\rho}).
\end{align*}

\end{document}
